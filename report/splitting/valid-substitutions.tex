\clearpage

\section*{Valid Substitutions}

We wish to work with sensible substitutions -- that map terms well-typed in one context to terms well-typed in another.
Hence we define a typing judgement for (concrete) substitutions:

\begin{judgement}{\subHasCtx{\ctx'}{\sub}{\ctx}}
{$\sub$ is a valid substitution, matching $\ctx$, relative to $\ctx'$ (and $\sig$)}
%
\begin{prooftree}
  \leftl{\rulename{S-Nil} :}
  \ax{\subHasCtx{\ctx'}{\emptysub}{\nil}}
\end{prooftree}

\begin{prooftree}
  \ninf{\subHasCtx{\ctx'}{\sub}{\ctx}}
  \ninf{\hasType{\ctx'}{M}{P[\sub]}}
  \leftl{\rulename{S-Cons} :}
  \binf{\subHasCtx{\ctx'}{(\sub, \for{M}{x})}{(\ctx, \typing{x}{P})}}
\end{prooftree}
%
\end{judgement}

Given $\ctx$ and $\sub$, we say that $\sub$ matches $\ctx$ if there exists $\ctx'$ such that $\subHasCtx{\ctx'}{\sub}{\ctx}$.
If $\sub$ matches $\ctx$, we have in particular $\dom{[\sub]} \subseteq \dom{\ctx}$.
For any valid context $\ctx$, we define $\idsub_{\ctx}$ to be the identity substitution with explicit bindings for all $x \in \dom{\ctx}$ in order.
Then clearly $\subHasCtx{\ctx}{\idsub_{\ctx}}{\ctx}$.

% note that we may extend a substitution with identity bindings to match a larger context

The following lemma states the desired property that valid substitutions map terms well-typed in one context to terms well-typed in another.

\begin{lemma}[Valid substitutions preserve validity]
\label{lem:substitutions-preserve-validity}
Let $\sub$ be a substitution, $\ctx$ and $\ctx'$ contexts, $\ctx$ valid, $U$ and $V$ terms.
If $\subHasCtx{\ctx'}{\sub}{\ctx}$ (by $\E$) and $\hasType{\ctx}{U}{V}$ (by $\T$), then $\hasType{\ctx'}{U[\sub]}{V[\sub]}$ (by some $\T'$).
\end{lemma}

\begin{proof}
% We only show the cases where $U$ is an object and $V$ a type; the remaining cases are analogous.
Omitted for now.
\end{proof}

Note that we do not require the contexts in the judgement $\subHasCtx{\ctx'}{\sub}{\ctx}$ to be valid.
And given $\sub$ and $\ctx$, there clearly may be infinitely many choices for the context $\ctx'$.
Fortunately, as the following lemmas attest, there is always a ``least'' $\ctx'$ which is in fact valid.

\begin{lemma}
\label{lem:minimal-codomain-object}
Let $\ctx$ and $\ctxvarvar$ be contexts, $\ctx$ valid (but $\ctxvarvar$ not necessarily so), such that $\ctx \subseteq \ctxvarvar$.
Let $M$ be an object and $P$ a type such that $\isType{\ctx}{P}$, and suppose $\hasType{\ctxvarvar}{M}{P}$ (by $\T$).
Then there exists $\ctx'$ with $\dom{\ctx} \cap \dom{\ctx'} = \emptyset$ such that $(\ctx, \ctx')$ is valid, $(\ctx, \ctx') \subseteq \ctxvarvar$ and $\hasType{(\ctx, \ctx')}{M}{P}$ (by some $\T'$).
\end{lemma}

\begin{proof}
By induction on $\T$, case analysis on $M$.

\begin{itemize}
	\item Case $M = x$ \\
  We have
  %
  \begin{prooftree}
  \ninf{$(\typing{x}{P}) \in \ctxvarvar$}
  \leftl{$\T =$}
  \uinf{\hasType{\ctxvarvar}{x}{P}}
  \end{prooftree}
  %
  If $x \in \dom{\ctx}$, we must have $(\typing{x}{P}) \in \ctx$ (since $\ctx \subseteq \ctxvarvar$), and so we can take $\ctx' = \nil$ and construct $\T'$ using rule \rulename{M-Var}.
  Otherwise, take $\ctx' = \typing{x}{P}$ (noting $\isType{\ctx}{P}$).
    
  \item Case $M = c \; M_1 \; \ldots \; M_n$ \\
  So $\T$ contains a subderivation $\I$ of $\typing{c}{(\Pi \; \typing{x_1}{P_1} \; \ldots \; \Pi \; \typing{x_n}{P_n} \, . \; P_c)} \in \sig$ and subderivations $\T_i$, for $i = 1, \ldots, n$, of $\hasType{\ctxvarvar}{M_i}{P_i[\for{M_1}{x_1}]\ldots[\for{M_{i-1}}{x_{i-1}}]}$.
  And $P = P_c[\for{M_1}{x_1}]\ldots[\for{M_n}{x_n}]$.
  
  Noting $\isType{\ctx}{P_1}$ (since $P_1$ must be ground by validity of $\sig$), by the IH on $\T_1$ with $\ctx$, we get $\T_1'$ showing $\hasType{(\ctx, \ctx_1)}{M_1}{P_1}$ for some $\ctx_1$ such that $(\ctx, \ctx_1)$ is valid and $(\ctx, \ctx_1) \subseteq \ctxvarvar$.
  Then, noting $\isType{(\ctx, \ctx_1)}{P_2[\for{M_1}{x_1}]}$ (by $\T_1'$ and the validity of $\sig$), by the IH on $\T_2$ with $(\ctx, \ctx_1)$, we get $\T_2'$ showing $\hasType{(\ctx, \ctx_1, \ctx_2)}{M_2}{P_2[\for{M_1}{x_1}]}$ for some $\ctx_2$ such that $(\ctx, \ctx_1, \ctx_2)$ is valid and $(\ctx, \ctx_1, \ctx_2) \subseteq \ctxvarvar$.
  And so on, formally by mathematical induction on $n$, until finally we get $\T_n'$ showing $\hasType{(\ctx, \ctx_1, \ldots, \ctx_n)}{M_n}{P_n[\for{M_1}{x_1}]\ldots[\for{M_{n-1}}{x_{n-1}}]}$ for some $\ctx_n$ such that $(\ctx, \ctx_1, \ldots, \ctx_n)$ is valid and $(\ctx, \ctx_1, \ldots, \ctx_n) \subseteq \ctxvarvar$.
  
  Now take $\ctx' = (\ctx_1, \ldots, \ctx_n)$.
  Weakening $\T_i'$ to show $\hasType{(\ctx, \ctx_1, \ldots, \ctx_n)}{M_i}{P_i[\for{M_1}{x_1}]\ldots[\for{M_{i-1}}{x_{i-1}}]}$, for $i = 1, \ldots, n-1$, we can use these along with $\I$ to construct a derivation of $\hasType{(\ctx, \ctx_1, \ldots, \ctx_n)}{c \; M_1 \; \ldots \; M_n}{P_c[\for{M_1}{x_1}]\ldots[\for{M_n}{x_n}]}$ as required.
\end{itemize}
%
\end{proof}

\begin{lemma}
\label{lem:minimal-codomain-substitution}
Let $\ctx$, $\ctxvar$ and $\ctxvarvar$ be contexts, $\ctx$ and $\ctxvar$ valid (but $\ctxvarvar$ not necessarily so), such that $\ctx \subseteq \ctxvarvar$.
Let $\sub$ be a substitution for which $\subHasCtx{\ctxvarvar}{\sub}{\ctxvar}$ (by $\E$).
Then there exists $\ctx'$ with $\dom{\ctx} \cap \dom{\ctx'} = \emptyset$ such that $(\ctx, \ctx')$ is valid, $(\ctx, \ctx') \subseteq \ctxvarvar$ and $\subHasCtx{(\ctx, \ctx')}{\sub}{\ctxvar}$ (by some $\E'$).
\end{lemma}

\begin{proof}
By induction on $\E$.

\begin{itemize}
	\item Case \rulename{S-Nil} \\
  We have
  %
  \begin{prooftree}
    \leftl{$\E =$}
    \ax{\subHasCtx{\ctxvarvar}{\emptysub}{\nil}}
  \end{prooftree}
  %
  So $\sub = \emptysub$ and $\ctxvar = \nil$.
  Taking $\ctx' = \nil$, we clearly have the required $\subHasCtx{(\ctx, \nil)}{\emptysub}{\nil}$ by rule \rulename{S-Nil}.
  
  \item Case \rulename{S-Cons} \\
  We have
  %
  \begin{prooftree}
    \prem{\E_1}{\subHasCtx{\ctxvarvar}{\sub'}{\ctxvar'}}
    \prem{\T}{\hasType{\ctxvarvar}{M}{P[\sub']}}
    \leftl{$\E =$}
    \binf{\subHasCtx{\ctxvarvar}{(\sub', \for{M}{x})}{(\ctxvar', \typing{x}{P})}}
  \end{prooftree}
  %
  So $\sub = (\sub', \for{M}{x})$ and $\ctxvar = (\ctxvar', \typing{x}{P})$.
  
  By the IH on $\E_1$ with $\ctx$, we get $\E_1'$ showing $\subHasCtx{(\ctx, \ctx_1)}{\sub'}{\ctxvar'}$ for some $\ctx_1$ such that $(\ctx, \ctx_1)$ is valid and $(\ctx, \ctx_1) \subseteq \ctxvarvar$.
  
  Note that we have $\isType{\ctxvar'}{P}$, and so $\hasType{\ctxvar'}{P}{\type}$ (say by $\K$), by validity of $\ctxvar$.
  And by Lemma~\ref{lem:substitutions-preserve-validity} on $\E_1'$ and $\K$, we get $\hasType{(\ctx, \ctx_1)}{P[\sub']}{\type}$ and so $\isType{(\ctx, \ctx_1)}{P[\sub']}$.
  Then by Lemma~\ref{lem:minimal-codomain-object} on $\T$ with $(\ctx, \ctx_1)$, we get $\T'$ showing $\hasType{(\ctx, \ctx_1, \ctx_2)}{M}{P[\sub']}$ for some $\ctx_2$ such that $(\ctx, \ctx_1, \ctx_2)$ is valid and $(\ctx, \ctx_1, \ctx_2) \subseteq \ctxvarvar$.
  
  Now take $\ctx' = (\ctx_1, \ctx_2)$, and weaken $\E_1'$ to show $\subHasCtx{(\ctx, \ctx_1, \ctx_2)}{\sub'}{\ctxvar'}$.
  Using $\E_1'$ and $\T'$, we can then construct the required derivation of $\subHasCtx{(\ctx, \ctx_1, \ctx_2)}{(\sub', \for{M}{x})}{(\ctxvar', \typing{x}{P})}$ by rule \rulename{S-Cons}.
\end{itemize}
%
\end{proof}

\begin{lemma}
\label{lem:minimal-codomain}
Let $\ctxvar$ be a valid context, $\sub$ a substitution.
If $\sub$ matches $\ctxvar$, then there exists a valid $\ctx$ such that $\subHasCtx{\ctx}{\sub}{\ctxvar}$.
Moreover, for any $\ctx'$ satisfying $\subHasCtx{\ctx'}{\sub}{\ctxvar}$, we have $\ctx \subseteq \ctx'$.
\end{lemma}

\begin{proof}
Since $\sub$ matches $\ctxvar$, we have $\subHasCtx{\ctx'}{\sub}{\ctxvar}$ (say by $\E$) for some $\ctx'$.
By Lemma~\ref{lem:minimal-codomain-substitution} on $\E$ with $\nil$, we get a valid $\ctx$ such that $\subHasCtx{\ctx}{\sub}{\ctxvar}$.
We also get that $\ctx \subseteq \ctx'$, and this is sufficient since $\ctx'$ was arbitrary.
\end{proof}

Clearly, given matching $\sub$ and $\ctxvar$, a valid context $\ctx$ of Lemma~\ref{lem:minimal-codomain} is unique up to ordering.
% The proofs of Lemma~\ref{lem:canon-codomain-object} and \ref{lem:canon-codomain-substitution}, however, together describe an algorithm for constructing a context with a specific order.
% The variables in $\dom{\ctx}$ are ordered by their (first) occurrences in the sequence of objects $(M_1, \ldots, M_n)$, for $\sub = (\for{M_1}{x_1}, \ldots, \for{M_n}{x_n})$.
% We call this unique context $\ctx$ the \emph{canonical codomain} of $\sub$.
We call such a context a \emph{minimal codomain} of $\sub$.
% should probably be a definition

Finally, we define composition of concrete substitutions, denoted by $\circ$ in order to distinguish it from composition of abstract substitutions, inductively:
\begin{align*}
\comp{\emptysub}{\subvar} &\mdefi \emptysub \\
\comp{(\sub, \for{M}{x})}{\subvar} &\mdefi (\comp{\sub}{\subvar}, \for{M[\subvar]}{x})
\end{align*}

This definition is slightly confusing when considered in relation to composition of abstract substitution; note, in general, $[\comp{\emptysub}{\subvar}] = [\emptysub] \neq [\subvar] = [\emptysub][\subvar]$.
%We will, however, only compose concrete substitutions for which the two notions of composition coincide, i.e. for which $[\comp{\sub}{\subvar}] = [\sub][\subvar]$.
% that is not quite correct, but perhaps we will only use it for terms T where T[\comp{\sub}{\subvar}] = T[\sub][\subvar]. note that instead.
% or perhaps we should require that dom([sub][subvar]) = dom([sub]). in that case, it might hold in general.

\begin{lemma}[Composition of concrete substitutions]
\label{lem:composition-concrete-substitutions}
Let $\sub$ and $\subvar$ be substitutions, $\ctx$, $\ctx'$ and $\ctx''$ contexts, the former two valid.
If $\subHasCtx{\ctx'}{\sub}{\ctx}$ (by $\E$) and $\subHasCtx{\ctx''}{\subvar}{\ctx'}$ (by $\E'$), then $\subHasCtx{\ctx''}{\comp{\sub}{\subvar}}{\ctx}$ (by some $\E''$).
\end{lemma}

\begin{proof}
Omitted for now.
\end{proof}

%%% end
