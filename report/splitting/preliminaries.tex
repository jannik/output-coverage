\clearpage

\section*{Preliminaries}

\subsection*{LF Syntax}

We recall our limited LF syntax:
\begin{align*}
  K &\defi \type \alt \pi{\typing{x}{P}}{K} \\
  P &\defi a \alt \app{P}{M} \\
  A &\defi P \alt \pi{\typing{x}{P}}{A} \\
  N &\defi c \alt \app{N}{M} \\
  M &\defi N \alt x
\end{align*}
So $K$ will denote kinds, $P$ primitive types, $A$ general types, $N$ composite objects, and $M$ general objects.
And $T$ will occasionally denote a term, i.e. anything allowed in the above categories.
We will also use the meta-variable $U$ to range over both types and objects ($A$ and $M$) and similarly $V$ for kinds and types ($K$ and $A$).
The meta-variables $a$, $c$ and $x$ (also $y$) range over some implicit sets of symbols -- of type constants, object constants and variables, respectively.

We employ the common distinction between \emph{bound} and \emph{free} variable occurrences in a term.
For any term $T$, let $\FV{T}$ denote the set of variables that occur free in $T$.
A term $T$ is said to be \emph{ground} if it contains only bound variable occurrences, i.e. if $\FV{T} = \emptyset$.



\subsection*{Substitution}

We need a notion of substitution, and it will be convenient to distinguish between two related concepts -- abstract and concrete substitutions.

\begin{definition}[Abstract substitution]
\label{def:abstract-substitution}
An \emph{abstract substitution} is a mapping from variables to objects; from the set of all variable symbols to the syntactic category of objects.
It is required that such a mapping is the identity on all but a finite subset of variables known as its \emph{support}.
(We denote abstract substitutions with the meta-variables $\abssub$ and $\abssubvar$.)
\end{definition}

By convention, application of substitutions is written in suffix notation, i.e. $x \abssub$ instead of $\abssub(x)$.
We define
\begin{align*}
  \supp{\abssub} &\mdefi \set{x \| x \abssub \neq x} \\
  \vran{\abssub} &\mdefi \bigcup{\set{\FV{x \abssub} \| x \in \supp{\abssub}}}
\end{align*}

Since the support of an abstract substitution $\abssub$ must be finite, say $\supp{\abssub} = \set{x_1, \ldots, x_n}$, we can unambiguously represent $\abssub$ by a set of mappings $\set{x_1 \mapsto x_1 \abssub, \ldots, x_n \mapsto x_n \abssub}$.
Note, however, that such a set does not represent an abstract substitution if it contains multiple mappings for any one variable, i.e. it must represent a (partial) function.
$\emptyset$ is then the identity substitution.

Composition of abstract substitutions is denoted by juxtaposition, i.e. $x (\abssubvar \abssub) \mdefi (x \abssubvar) \abssub$ for all $x$.
(Due to the suffix notation, composition here is naturally read left-to-right as opposed to regular function composition.)
The restriction of an abstract substitution $\abssub$ to a set of variables $X$ (typically intersecting non-trivially with its support), written $\restr{\abssub}{X}$, is the abstract substitution $\set{x \mapsto x \abssub \| x \in \supp{\abssub} \cap X}$.

Abstract substitutions are tacitly lifted to act on terms (objects, types or kinds) in a capture-avoiding manner, e.g. for types:
\[ A \abssub \mdefi
  \begin{cases}
    a \; M_1 \abssub \; \ldots \; M_n \abssub           & \text{if $A = a \; M_1 \; \ldots \; M_n$} \\
    \pi{\typing{y}{P \abssub}}{(A' \abssubvar) \abssub} & \text{if $A = \pi{\typing{x}{P}}{A'}$,} \\
                                                        & \text{letting $\abssubvar = \set{x \mapsto y}$ for some $y \notin \supp{\abssub} \cup \vran{\abssub} \cup \FV{A}$}
  \end{cases}
\]
% define inductively instead? and argue that it isn't a problem that (A' \abssubvar) is not a subterm of A!
% maybe note that this definition might rename unnecessarily in cases where x \notin \supp{\abssub} \cup \vran{\abssub}

\begin{definition}[Concrete substitution]
\label{def:concrete-substitution}
A \emph{(concrete) substitution} is an ordered list of pairs of variables and objects, using the notation $\for{M}{x}$ (read as ``$M$ for $x$'') for the pair $\tup{x, M}$:
\[ \sub, \subvar \defi \emptysub \alt \sub, \for{M}{x} \]
\end{definition}

$\emptysub$ denotes the empty list, but we will occasionally omit the symbol when listing a concrete substitution, e.g. $(\for{M_x}{x}, \for{M_y}{y})$ instead of $(\emptysub, \for{M_x}{x}, \for{M_y}{y})$.
We require that any non-empty concrete substitution $(\sub, \for{M_x}{x})$ satisfies $x \neq y$ for all $\for{M_y}{y} \in \sub$.
% note the 'member' notation borrowed from set theory, maybe even define it (assuming it is first used here)
% consider defining the domain/support of a concrete substitution to be \set{x \| \for{M}{x} \in \sub}

A concrete substitution $\sub = (\emptysub, \for{M_1}{x_1}, \ldots, \for{M_n}{x_n})$ naturally \emph{represents} a unique abstract substitution, written $[\sub]$, namely $\set{x_1 \mapsto M_1, \ldots, x_n \mapsto M_n}$.
We can ignore those $i$ for which $M_i = x_i$, since making identity mappings explicit has no effect on abstract substitutions.
Of course, an abstract substitution is in turn represented by many different concrete substitutions (in fact, with a support of $n$ variables there will be $n!$ different ways of ordering the list, and adding explicit identity pairs for variables outside the support leads to an infinite number of substitutions).



\subsection*{Signatures and Contexts}

We wish to work with terms that are ``valid'' relative to a signature and a context.
Signatures ($\sig$) associate constants with types and kinds, and contexts ($\ctx$) associative variables with primitive types:
\begin{align*}
  \sig &\defi \nil \alt \sig, \typing{a}{K} \alt \sig, \typing{c}{A} \\
  \ctx, \ctxvar, \ctxvarvar &\defi \nil \alt \ctx, \typing{x}{P}
\end{align*}

Here, $\nil$ denotes the empty list.
We require that any non-empty context $(\ctx, \typing{x}{P_x})$ satisfies $x \neq y$ for all $\typing{y}{P_y} \in \ctx$.
Similarly, any signature of the form $(\sig, \typing{a}{K})$ must satisfy $a \neq a'$ for all $\typing{a'}{K'} \in \sig$ -- and analogously for signatures of the form $(\sig, \typing{c}{A})$.
That is, contexts and signatures must be (partial) functions.

The \emph{domain} of a context $\ctx = (\nil, \typing{x_1}{P_1}, \ldots, \typing{x_n}{P_n})$, written $\dom{\ctx}$, is the set of variables $\set{x_1, \ldots, x_n}$.
Given a context $\ctx$ and a set of variables $X$, typically $X \subseteq \dom{\ctx}$, we let $\ctx \setminus X$ be the context $\ctx$ with all assignments $\typing{x}{P}$ for $x \in X$ removed.
We will borrow further notation from set theory and write $\ctx \subseteq \ctx'$ when $\ctx$ is a subset of $\ctx'$, both viewed as sets of assignments (i.e. disregarding the orders).

Concatenation of two contexts will, by abuse of notation, be denoted with a comma.
That is, if $\ctx = (\nil, \typing{x_1}{P_1}, \ldots, \typing{x_m}{P_m})$ and $\ctx' = (\nil, \typing{x_1'}{P_1'}, \ldots, \typing{x_n'}{P_n'})$ with $\dom{\ctx} \cap \dom{\ctx'} = \emptyset$ then $(\ctx, \ctx') \mdefi (\nil, \typing{x_1}{P_1}, \ldots, \typing{x_m}{P_m}, \typing{x_1'}{P_1'}, \ldots, \typing{x_n'}{P_n'})$.

A term $T$ is \emph{ground with respect to} a context $\ctx$ if $\FV{T} \subseteq \dom{\ctx}$.
Clearly, $T$ is ground (as previously defined) if it is ground with respect to the empty context.

Letting $\ctx = (\nil, \typing{x_1}{P_1}, \ldots, \typing{x_n}{P_n})$, we will occasionally write $\pi{\ctx}{P}$ as an economical shorthand for $\Pi \, \typing{x_1}{P_1} \, \ldots \, \Pi \, \typing{x_n}{P_n} . \, P$.
And for a constant $c$, $\app{c}{\ctx}$ is shorthand for the object $c \; x_1 \; \ldots \; x_n$.

Having introduced contexts, we again lift abstract substitutions to act on contexts; $\ctx \abssub$ is the result of applying $\abssub$ to each primitive type in $\ctx$.

\subsection*{LF Type System}

We are now ready to present the type system of our limited LF.

%First, we need a judgement for looking up the type of a variable in a context -- and analogous judgements for lookup in signatures.
%
%\begin{judgement}{$(\typing{x}{P}) \in \ctx$}
%{$x$ is assigned type $P$ in $\ctx$}
%%
%\begin{prooftree}
  %\leftl{\rulename{L-Here} :}
  %\ax{$(\typing{x}{P}) \in (\ctx, \typing{x}{P})$}
%\end{prooftree}
%
%\begin{prooftree}
  %\ninf{$(\typing{x}{P}) \in \ctx$}
  %\leftl{\rulename{L-There} :}
  %\rightl{($x' \neq x$)}
  %\uinf{$(\typing{x}{P}) \in (\ctx, \typing{x'}{P'})$}
%\end{prooftree}
%%
%\end{judgement}

\begin{judgement}{\hasType{\ctx}{M}{A}}
{$M$ has type $A$ relative to $\ctx$ (and $\sig$)}
%
\begin{prooftree}
  \ninf{$(\typing{c}{A}) \in \sig$}
  \leftl{\rulename{M-Const} :}
  \uinf{\hasType{\ctx}{c}{A}}
\end{prooftree}

\begin{prooftree}
  \ninf{$(\typing{x}{P}) \in \ctx$}
  \leftl{\rulename{M-Var} :}
  \uinf{\hasType{\ctx}{x}{P}}
\end{prooftree}

\begin{prooftree}
  \ninf{\hasType{\ctx}{N}{(\pi{\typing{x}{P}}{A}})}
  \ninf{\hasType{\ctx}{M}{P}}
  \leftl{\rulename{M-App} :}
  \binf{\hasType{\ctx}{\app{N}{M}}{A[\for{M}{x}]}}
\end{prooftree}
%
\end{judgement}

\begin{judgement}{\hasKind{\ctx}{P}{K}}
{$P$ has kind $K$ relative to $\ctx$ (and $\sig$)}
%
\begin{prooftree}
  \ninf{$(\typing{a}{K}) \in \sig$}
  \leftl{\rulename{P-Atom} :}
  \uinf{\hasKind{\ctx}{a}{K}}
\end{prooftree}

\begin{prooftree}
  \ninf{\hasKind{\ctx}{P}{(\pi{\typing{x}{P'}}{K}})}
  \ninf{\hasType{\ctx}{M}{P'}}
  \leftl{\rulename{P-App} :}
  \binf{\hasKind{\ctx}{\app{P}{M}}{K[\for{M}{x}]}}
\end{prooftree}
%
\end{judgement}

\begin{judgement}{\isType{\ctx}{A}}
{$A$ is a valid type relative to $\ctx$ (and $\sig$)}
%
\begin{prooftree}
  \ninf{\hasKind{\ctx}{P}{\type}}
  \leftl{\rulename{A-Prim} :}
  \uinf{\isType{\ctx}{P}}
\end{prooftree}

\begin{prooftree}
  \ninf{\isType{\ctx}{P}}
  \ninf{\isType{\ctx, \typing{x}{P}}{A}}
  \leftl{\rulename{A-Pi} :}
  \rightl{($x \notin \dom{\ctx}$)}
  \binf{\isType{\ctx}{\pi{\typing{x}{P}}{A}}}
\end{prooftree}
%
\end{judgement}

\begin{judgement}{\isKind{\ctx}{K}}
{$A$ is a valid kind relative to $\ctx$ (and $\sig$)}
%
\begin{prooftree}
  \leftl{\rulename{K-Type} :}
  \ax{\isKind{\ctx}{\type}}
\end{prooftree}

\begin{prooftree}
  \ninf{\isType{\ctx}{P}}
  \ninf{\isKind{\ctx, \typing{x}{P}}{K}}
  \leftl{\rulename{K-Pi} :}
  \rightl{($x \notin \dom{\ctx}$)}
  \binf{\isKind{\ctx}{\pi{\typing{x}{P}}{K}}}
\end{prooftree}
%
\end{judgement}

\begin{judgement}{\isSig{\sig}}
{$\sig$ is a valid signature}
%
\begin{prooftree}
  \leftl{\rulename{Sig-Nil} :}
  \ax{\isSig{\nil}}
\end{prooftree}

\begin{prooftree}
  \ninf{\isSig{\sig}}
  \ninf{\isKind{\nil}{K}}
  \leftl{\rulename{Sig-Kind} :}
  \binf{\isSig{\sig, \typing{a}{K}}}
\end{prooftree}

\begin{prooftree}
  \ninf{\isSig{\sig}}
  \ninf{\isType{\nil}{A}}
  \leftl{\rulename{Sig-Type} :}
  \binf{\isSig{\sig, \typing{c}{A}}}
\end{prooftree}
%
\end{judgement}

\begin{judgement}{\isCtx{\ctx}}
{$\ctx$ is a valid context (relative to $\sig$)}
%
\begin{prooftree}
  \leftl{\rulename{Ctx-Nil} :}
  \ax{\isCtx{\nil}}
\end{prooftree}

\begin{prooftree}
  \ninf{\isCtx{\ctx}}
  \ninf{\isType{\ctx}{P}}
  \leftl{\rulename{Ctx-Cons} :}
  \rightl{($x \notin \dom{\ctx}$)}
  \binf{\isCtx{\ctx, \typing{x}{P}}}
\end{prooftree}
%
\end{judgement}

In particular, the types of a valid context are ground with respect to the previous assignments in the context.
It should be noted that adding assignments to or removing assignments from valid contexts will not in general preserve validity.
And $\ctx \abssub$ is generally not a valid context either, even if $\ctx$ is valid.
Indeed, we will need to be careful about whether or not a given context is valid.

Clearly, if $\ctx$ and $\ctx'$ are both valid, then the concatenation $(\ctx, \ctx')$ is a valid context.
% note: we will use this result tacitly

% consider defining an explicit weakening lemma for typing derivations

We will, however, only deal with signatures that are valid:
In the following, we will take for granted the presence of a given valid signature $\sig$; all definitions and lemmas should really begin with ``Let $\sig$ be a valid signature.''

%%% end
