\clearpage

\section*{Unification}

In this section, we use the meta-variable $U$ to range over objects and primitive types (categories $M$ and $P$), $V$ for primitive types and the single primitive kind $\type$.
We also let $f$ and $g$ both range over constant symbols (i.e. object constants $c$ and type constants $a$).

\begin{definition}[Unification problem]
\label{def:unification-problem}
A \emph{unification problem} $\U$ is a finite multi-set of pairs of terms, limited to primitive types or objects.
For the elements of a unification problem, called \emph{equations}, we use the notation $U \dot{=} U'$ instead of $\tup{U, U'}$.
That is, $\U = \set{\ueqn{U_1}{U_1'}, \ldots, \ueqn{U_n}{U_n'}}$.
\end{definition}

% maybe save a proper definition for ordered unification problems, or remove the need for ordering altogether!

We extend a few definitions to apply also to unification problems:
\begin{align*}
\FV{\U} &\mdefi \bigcup{\set{\FV{U} \cup \FV{U'} \| \ueqn{U}{U'} \in \U}} \\
\U \abssub &\mdefi \set{\ueqn{U \abssub}{U' \abssub} \| \ueqn{U}{U'} \in \U}
\end{align*}

In the following, we will suppose that a unification problem always comes equipped with an ordering on the multi-set.
This will be useful later, but the ordering has no effect on whether a problem has a solution -- as defined momentarily.
Hence
\[ \U \defi \cdot \alt \ueqn{U}{U'}, \U \]

\begin{definition}[Unifier, mgu]
\label{def:unifier-mgu}
An abstract substitution $\abssub$ is a \emph{unifier} of (and is said to \emph{unify}) an equation $\ueqn{U}{U'}$ if we have $U \abssub = U' \abssub$.
It is a \emph{most general unifier} (or \emph{mgu}) of the equation if, for every unifier $\abssubvar$ of the equation, there exists an abstract substitution $\abssub'$ such that $\abssub \abssub' = \abssubvar$.

The definitions extend to sets of equation; $\abssub$ is a unifier of a set of equations if it unifies each equation in the set.
For a unification problem $\U$, we say that $\abssub$ is a \emph{solution} of $\U$ if it unifies all $(\ueqn{U}{U'}) \in \U$.
\end{definition}

Given a unification problem $\U$, we let $\mgu{\U}$ denote the (potentially empty) set of most general unifiers $\abssub$ that are idempotent, i.e. $\abssub \abssub = \abssub$, and
satisfy $\supp{\abssub} \cup \vran{\abssub} \subseteq \FV{\U}$. Note that idempotence implies $\supp{\abssub} \cap \vran{\abssub} = \emptyset$.
% maybe subseteq should really be eq
% mention something about the properties of the first-order case, e.g. all mgus are instances of each other, my defined mgu set is finite..

% reference prop 4.10 in old unification article?
% It is decidable whether a unification problem is unifiable, and indeed the common algorithms produce a mgu with the above properties (if the problem is unifiable).

\subsubsection*{An Inference System for Unification}

In order to reason about unification, we present a simple formalisation that captures the common algorithms and their properties.
This is essentially equivalent to the common inference system (inconveniently named ``$\U$'') presented by e.g. Baader and Snyder [ref].
The correctness proof below is adapted from their book.
% ref: Baader, F., and W. Snyder, "Unification Theory," chapter eight of Handbook of Automated Deduction, Springer Verlag, Berlin (2001).

A \emph{unification state} $\ustate$ is either $\bot$ (representing failure) or a pair consisting of an ordered unification problem and an abstract substitution:
\[ \ustate \defi \bot \alt \tup{\U \| \abssub} \]
We now define a transition relation $\Longrightarrow$ on unification states, which is naturally extended to a multi-step relation $\Longrightarrow^*$ (the reflexive-transitive closure).

\begin{judgement}{\ustep{\ustate}{\ustate'}}
{$\ustate$ steps to $\ustate'$} % improve
%
\begin{itemize}
	\item[] \rulename{Triv}:
    \ax{\ustep{\tup{\ueqn{U}{U}, \U \| \abssub}}{\tup{\U \| \abssub}}}
    \DisplayProof
  \item[] \rulename{Decomp}:
    \ax{\ustep{\tup{\ueqn{f \; M_1 \; \ldots \; M_n}{f \; M_1' \; \ldots \; M_n'}, \U \| \abssub}}{\tup{\ueqn{M_1}{M_1'}, \ldots, \ueqn{M_n}{M_n'}, \U \| \abssub}}}
    \DisplayProof
  \item[] \rulename{Clash}:
    \rightl{($f \neq g$ or $m \neq n$)}
    \ax{\ustep{\tup{\ueqn{f \; M_1 \; \ldots \; M_m}{g \; M_1' \; \ldots \; M_n'}, \U \| \abssub}}{\bot}}
    \DisplayProof
  \item[] \rulename{Orient}:
    \rightl{($M$ is not a variable)}
    \ax{\ustep{\tup{\ueqn{M}{x}, \U \| \abssub}}{\tup{\ueqn{x}{M}, \U \| \abssub}}}
    \DisplayProof
  \item[] \rulename{Occurs}:
    \rightl{($x \in \FV{M}$ but $x \neq M$)}
    \ax{\ustep{\tup{\ueqn{x}{M}, \U \| \abssub}}{\bot}}
    \DisplayProof
  \item[] \rulename{Elim}:
    \rightl{($x \notin \FV{M}$)}
    \ax{\ustep{\tup{\ueqn{x}{M}, \U \| \abssub}}{\tup{\U \set{x \mapsto M} \| \abssub \set{x \mapsto M}}}}
    \DisplayProof
\end{itemize}
%
\end{judgement}

\begin{lemma}[Properties]
\label{lem:unification-properties}
If $\usteps{\tup{\U \| \emptyset}}{\tup{\nil \| \abssub}}$, then
\begin{itemize}
  \item $\abssub$ satisfies $\supp{\abssub} \cup \vran{\abssub} \subseteq \FV{\U}$
  % in fact, eq instead of subseteq, but I only need subseteq
	\item $\abssub$ is idempotent, and so $\supp{\abssub} \cap \vran{\abssub} = \emptyset$
\end{itemize}
\end{lemma}

\begin{proof}
% only one rule adds to construction, and look at the way it does so
Omitted for now.
\end{proof}

\begin{lemma}[Soundness]
\label{lem:unification-soundness}
If $\usteps{\tup{\U \| \emptyset}}{\tup{\nil \| \abssub}}$, then $\abssub$ unifies every equation in $\U$.
\end{lemma}

\begin{proof}
Omitted for now.
\end{proof}

\begin{lemma}[Completeness]
\label{lem:unification-completeness}
If $\abssubvar$ unifies every equation in $\U$, then any maximal sequence of transformations from $\tup{\U \| \emptyset}$ must end in some $\tup{\nil \| \abssub}$ such that $\abssub \abssub' = \abssubvar$ for some $\abssub'$ (i.e. $\abssub$ is a most general unifier).
\end{lemma}

\begin{proof}
Omitted for now.
\end{proof}

\begin{lemma}[Correctness]
\label{lem:unification-correctness}
A unification problem $\U$ has a solution if and only if $\usteps{\tup{\U \| \emptyset}}{\tup{\nil \| \abssub}}$ is derivable for some $\abssub$, and in that case $\abssub \in \mgu{\U}$.
\end{lemma}

\begin{proof}
Immediate from Lemma~\ref{lem:unification-properties}, \ref{lem:unification-soundness} and \ref{lem:unification-completeness}.
\end{proof}

Furthermore, it is easy to derive from this inference system a deterministic and terminating algorithm that either reports failure or generates a most general unifier with the properties of Lemma~\ref{lem:unification-properties}.

Our motivation for introducing this formalisation, however, is to prove another property -- namely that well-typed unification problems have well-typed solutions.
% this seems to (wrongly?) suggest that all solutions will be well-typed, as opposed to just one such solution

\begin{definition}[Well-typed unification problems]
\label{def:well-typed-unification-problems}
An ordered unification problem $\U$ is \emph{well-typed} in a context $\ctx$ if either $\U = \nil$ or if $\U = (\ueqn{U}{U'}, \U')$ and both of the following hold:
\begin{itemize}
	\item For some type or kind $V$, we have $\hasType{\ctx}{U}{V}$ and $\hasType{\ctx}{U'}{V}$. That is, $U$ and $U'$ have the same type.
  \item For every substitution $\sub$ with $\subHasCtx{\ctx[\sub]}{\sub}{\ctx}$ such that $U[\sub] = U'[\sub]$, the unification problem $\U[\sub]$ is well-typed in $\ctx[\sub]$.
  % I'm not sure about this. can't \sub be something strange? no wait, I suppose it matches something and so preserves types..
\end{itemize}
\end{definition}


\begin{lemma}
\label{lem:unification-invariants}
Given a context $\ctx$, the following two conditions on unification states are together invariant under the transition relation.
We say that $\bot$ vacuously satisfies the conditions; they are defined for pairs $\tup{\U \| \abssub}$.
\begin{itemize}
	\item $\U$ is well-typed in $\ctx \abssub$.
  \item There exists $\sub$ with $\subHasCtx{\ctx[\sub]}{\sub}{\ctx}$ such that $[\sub] = \abssub$.
\end{itemize}
\end{lemma}

\begin{proof}
Suppose we have $\ustep{\ustate}{\ustate'}$ (by $\E$) with $\ustate$ satisfying the conditions.
We must show that $\ustate'$ satisfies the conditions as well.
By case analysis on $\E$; the only interesting cases are \rulename{Decomp} and \rulename{Elim}.
\begin{itemize}
	\item Case \rulename{Decomp}: \\
  We have $\ustate = \tup{\ueqn{f \; M_1 \; \ldots \; M_n}{f \; M_1' \; \ldots \; M_n'}, \U \| \abssub}$ and $\ustate' = \tup{\ueqn{M_1}{M_1'}, \ldots, \ueqn{M_n}{M_n'}, \U \| \abssub}$.
  It suffices to show that $(\ueqn{M_1}{M_1'}, \ldots, \ueqn{M_n}{M_n'}, \U)$ is well-typed in $\ctx \abssub$.
  Since by assumption $(\ueqn{f \; M_1 \; \ldots \; M_n}{f \; M_1' \; \ldots \; M_n'}, \U)$ is well-typed in $\ctx \abssub$, in particular we have a typing derivation
  %
  \begin{prooftree}
  \ninf{$\typing{f}{(\Pi \; \typing{x_1}{P_1} \; \ldots \; \Pi \; \typing{x_n}{P_n} \, . \; V)} \in \sig$}
  \uinf{\hasType{\ctx \abssub}{f}{(\Pi \; \typing{x_1}{P_1} \; \ldots \; \Pi \; \typing{x_n}{P_n} \, . \; V)}}
  \prem{\T_1}{\hasType{\ctx \abssub}{M_1}{P_1}}
  \binf{\hasType{\ctx \abssub}{f \; M_1}{(\Pi \; \typing{x_2}{P_2[\for{M_1}{x_1}]} \; \ldots \; \Pi \; \typing{x_n}{P_n[\for{M_1}{x_1}]} \, . \; V[\for{M_1}{x_1}])}}
  \noLine
  \uinf{$\vdots$}
  \noLine
  \uinf{\hasType{\ctx \abssub}{f \; M_1 \; \ldots \; M_{n-1}}{(\pi{\typing{x_n}{P_n \abssubvar}}{V \abssubvar}})}
  \prem{\T_n}{\hasType{\ctx \abssub}{M_n}{P_n \abssubvar}}
  \leftl{$\T =$}
  \binf{\hasType{\ctx \abssub}{f \; M_1 \; \ldots \; M_n}{V \abssubvar [\for{M_n}{x_n}]}}
  \end{prooftree}
  %
  (where $\abssubvar = [\for{M_1}{x_1}] \ldots [\for{M_{n-1}}{x_{n-1}}]$) and an analogous $\T'$ (containing $\T_1', \ldots, \T_n'$) showing $\hasType{\ctx \abssub}{f \; M_1' \; \ldots \; M_n'}{V [\for{M_1'}{x_1}] \ldots [\for{M_n'}{x_n}]}$.
  % note: we do not use in this case that the types are equal, i.e. that (V \abssubvar [\for{M_n}{x_n}]) == (V \abssubvar' [\for{M_n'}{x_n}]).
  % in fact, we could argue that they MUST be equal.
  
  By $\T_1$ and $\T_1'$, we have that $M_1$ and $M_1'$ have the same type.
  Now suppose there exists $\sub_1$ with $\subHasCtx{\ctx \abssub [\sub_1]}{\sub_1}{\ctx \abssub}$ such that $M_1[\sub_1] = M_1'[\sub_1]$.
  Applying $\sub_1$ to $M_2$ and $M_2'$, we get $\hasType{\ctx \abssub [\sub_1]}{M_2[\sub_1]}{P_2[\for{M_1}{x_1}][\sub_1]}$ and $\hasType{\ctx \abssub [\sub_1]}{M_2'[\sub_1]}{P_2[\for{M_1'}{x_1}][\sub_1]}$ (by $\T_2$ and $\T_2'$).
  And since $M_1[\sub_1] = M_1'[\sub_1]$, we clearly have $P_2[\for{M_1}{x_1}][\sub_1] = P_2[\for{M_1'}{x_1}][\sub_1]$ as required.
  
  Repeating this argument (formally by mathematical induction on $n$), we eventually get
  \[ \hasType{\ctx \abssub [\sub_1] \ldots [\sub_{n-1}]}{M_n[\sub_1] \ldots [\sub_{n-1}]}{P_n[\for{M_1}{x_1}] \ldots [\for{M_{n-1}}{x_{n-1}}][\sub_1] \ldots [\sub_{n-1}]} \]
  and
  \[ \hasType{\ctx \abssub [\sub_1] \ldots [\sub_{n-1}]}{M_n'[\sub_1] \ldots [\sub_{n-1}]}{P_n[\for{M_1'}{x_1}] \ldots [\for{M_{n-1}'}{x_{n-1}}][\sub_1] \ldots [\sub_{n-1}]} \]
  with
  \[ P_n[\for{M_1}{x_1}] \ldots [\for{M_{n-1}}{x_{n-1}}][\sub_1] \ldots [\sub_{n-1}] = P_n[\for{M_1'}{x_1}] \ldots [\for{M_{n-1}'}{x_{n-1}}][\sub_1] \ldots [\sub_{n-1}] \]
  
  Suppose now we have $\sub_n$ with $\subHasCtx{\ctx \abssub [\sub_1] \ldots [\sub_n]}{\sub_n}{\ctx \abssub [\sub_1] \ldots [\sub_{n-1}]}$ such that $M_n[\sub_1] \ldots [\sub_n] = M_n'[\sub_1] \ldots [\sub_n]$.
  Then by composition, we get $\sub \mdefi (\sub_1 \circ \ldots \circ \sub_n)$ with $\subHasCtx{\ctx \abssub [\sub_1] \ldots [\sub_n]}{\sub}{\ctx \abssub}$ and $M_i[\sub] = M_i'[\sub]$ for all $i$.
  % be careful with this composition. is it safe? do we need the composition lemma, which is currently defined only for VALID contexts?
  Clearly, $\sub$ also satisfies $(f \; M_1 \; \ldots \; M_n)[\sub] = (f \; M_1' \; \ldots \; M_n')[\sub]$.
  So, by assumption, $\U$ is well-typed in $\ctx \abssub [\sub] = \ctx \abssub [\sub_1] \ldots [\sub_n]$.
  And hence $(\ueqn{M_1}{M_1'}, \ldots, \ueqn{M_n}{M_n'}, \U)$ is well-typed in $\ctx \abssub$.
  % surely, you can manage a clearer proof once the definitions (of well-typed problems and of the invariants) are refined
  
  \item Case \rulename{Elim}: \\
  We have $\ustate = \tup{\ueqn{x}{M}, \U \| \abssub}$ with $x \notin \FV{M}$ and $\ustate' = \tup{\U \set{x \mapsto M} \| \abssub \set{x \mapsto M}}$.
  Since by assumption $(\ueqn{x}{M}, \U)$ is well-typed in $\ctx \abssub$, in particular we have $\hasType{\ctx \abssub}{M}{P_x}$ (by $\T$) where $(\typing{x}{P_x}) \in \ctx \abssub$.
  Extending the substitution $(\for{M}{x})$ with identity bindings to match $\ctx \abssub$, by $\T$ we then have $\subHasCtx{\ctx \abssub [\for{M}{x}]}{\sub}{\ctx \abssub}$ for some $\sub$ with $[\sub] = [\for{M}{x}] = \set{x \mapsto M}$.
  % is this unclear? I could introduce and prove a little lemma about extending a concrete substitution to match a context
  And naturally, noting $x \notin \FV{M}$, we have $x[\sub] = M[\sub]$.
  Then, by assumption, $\U \set{x \mapsto M} = \U[\sub]$ is well-typed in $\ctx (\abssub \set{x \mapsto M}) = \ctx \abssub \set{x \mapsto M} = \ctx \abssub [\sub]$.
  
  Finally, by assumption, there exists $\subvar$ with $\subHasCtx{\ctx[\subvar]}{\subvar}{\ctx}$ such that $[\subvar] = \abssub$.
  But then the composition $\subvar \circ \sub$ satisfies $\subHasCtx{\ctx[\subvar \circ \sub]}{\subvar \circ \sub}{\ctx}$ and $[\subvar \circ \sub] = \abssub \set{x \mapsto M}$ as required.
  And so both conditions hold for $\ustate'$.
  % if the composition lemma doesn't hold for non-valid substitutions (IT DOES NOT), I need to refine this lemma.
\end{itemize}
\end{proof}

\begin{lemma}
\label{lem:unification-existence}
If a unification problem $\U$ well-typed in $\ctx$ has a solution, then there exists $\sub$ with $\subHasCtx{\ctx[\sub] \setminus \supp{[\sub]}}{\sub}{\ctx}$ such that $[\sub] \in \mgu{\U}$.
\end{lemma}

\begin{proof}
Suppose $\U$ has a solution.
Then we know that $\usteps{\tup{\U \| \emptyset}}{\tup{\nil \| \abssub}}$ is derivable for some $\abssub \in \mgu{\U}$.
% by completeness!
Since $\U$ is well-typed in $\ctx \emptyset = \ctx$, and $\idsub_{\ctx}$ satisfies $\subHasCtx{\ctx[\idsub_{\ctx}]}{\idsub_{\ctx}}{\ctx}$ and $[\idsub_{\ctx}] = \emptyset$, the two conditions of Lemma~\ref{lem:unification-invariants} are satisfied for $\tup{\U \| \emptyset}$.
Hence, by Lemma~\ref{lem:unification-invariants}, there exists $\sub$ with $\subHasCtx{\ctx[\sub]}{\sub}{\ctx}$ such that $[\sub] = \abssub$ (and so $[\sub] \in \mgu{\U}$ as required).

And since we have $\supp{[\sub]} \cap \vran{[\sub]} = \emptyset$, certainly $\subHasCtx{\ctx[\sub]}{\sub}{\ctx}$ implies $\subHasCtx{\ctx[\sub] \setminus \supp{[\sub]}}{\sub}{\ctx}$ (the same derivation suffices).
\end{proof}

\begin{lemma}
\label{lem:unification-mgu}
Let $\ctx$ and $\ctxvar$ be valid contexts such that $\ctxvar \subseteq \ctx[\sub] \setminus \supp{[\sub]}$.
Let $\ctx'$ be a (not necessarily valid) context, $\sub$ and $\subvar$ concrete substitutions such that $\subHasCtx{\ctxvar}{\sub}{\ctx}$ and $\subHasCtx{\ctx'}{\subvar}{\ctx}$.
Let $\U$ be a unification problem, and suppose $[\sub] \in \mgu{\U}$ and $[\subvar]$ a unifier of the equation in $\U$.
Then there exists a concrete substitution $\sub'$ such that $[\subvar] = [\sub][\sub']$ and $\subHasCtx{\ctx'}{\sub'}{\ctxvar}$.
% should probably be changed to ``.. such that $[\subvar] = [\sub \circ \sub']$ ..''
\end{lemma}

\begin{proof}
Since $[\sub]$ is a most general unifier of the elements of $\U$, there exists an abstract substitution $\abssub$ such that $[\subvar] = [\sub] \abssub$.
We show by structural induction on $\ctxvar$ that there exists a concrete substitution $\sub'$ such that $\subHasCtx{\ctx'}{\sub'}{\ctxvar}$ and which agrees with $\abssub$ on all $x \in \dom{\ctxvar}$.
\begin{itemize}
	\item Case $\ctxvar = (\nil)$ \\
  Take $\sub' = \emptysub$; the empty substitution satisfies $\subHasCtx{\ctx'}{\emptysub}{\nil}$ (by rule \rulename{S-Nil}) and vacuously agrees with $\abssub$ on all $x \in \dom{\nil} = \emptyset$.
  \item Case $\ctxvar = (\ctxvar', \typing{x}{P_x[\sub]})$ \\
  By the induction hypothesis on $\ctxvar'$, there exists $\sub''$ with $\subHasCtx{\ctx'}{\sub''}{\ctxvar'}$ (by some $\E$) and $[\sub''] = \restr{\abssub}{\dom{\ctxvar'}}$.
  Now take $\sub' = (\sub'', \for{x[\subvar]}{x})$.
  Since $x \notin \supp{[\sub]}$ (by definition of $\ctxvar$), we have $x \abssub = x [\sub] \abssub = x[\subvar]$, and so $[\sub'] = \restr{\abssub}{\dom{\ctxvar}}$ as required.
  
  We have $\subHasCtx{\ctx'}{\subvar}{\ctx}$ by assumption, and any derivation of this must contain a derivation $\T$ of $\hasType{\ctx'}{x[\subvar]}{P_x[\subvar]}$ (since $x \in \dom{\ctx}$).
  Finally, noting $P_x[\sub][\sub''] = P_x[\sub]\abssub = P_x[\subvar]$ (since $\FV{P_x[\sub]} \subseteq \dom{\ctxvar'}$), we construct a derivation of $\subHasCtx{\ctx'}{\sub'}{\ctxvar}$.
  %
  \begin{prooftree}
  \prem{\E}{\subHasCtx{\ctx'}{\sub''}{\ctxvar'}}
  \prem{\T}{\hasType{\ctx'}{x[\subvar]}{P_x[\sub][\sub'']}}
  \binf{\subHasCtx{\ctx'}{(\sub'', \for{x[\subvar]}{x})}{(\ctxvar', \typing{x}{P_x[\sub]})}}
  \end{prooftree}
  %
\end{itemize}
Hence there exists $\sub'$ such that $\subHasCtx{\ctx'}{\sub'}{\ctxvar}$ and $[\sub'] = \restr{\abssub}{\dom{\ctxvar}}$.
Since $\supp{\abssub} \subseteq \dom{\ctx}$, $\sub'$ and $\abssub$ agree everywhere outside of $\dom{\ctx}$ as well.
And since $\supp{[\sub]} \cap \vran{[\sub]} = \emptyset$, agreement is irrelevant for $\supp{[\sub]}$.
So $[\sub][\sub'] = [\sub]\abssub = [\subvar]$.
\end{proof}

%%% end
