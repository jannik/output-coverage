\clearpage

\section*{Operational Semantics}

Given a valid signature, we can interpret it as a logic program.
We will soon define logic programs as being valid with respect to a valid signature, but there is a canonical way to derive from a signature a largest such program.

Recall the syntactic categories of terms, in particular
\[ \text{Primitive types} \qquad P, Q \defi  a \alt \app{P}{M} \]
We define \textit{goals} (not to be confused with coverage goals), \textit{clauses}, and \textit{logic programs} as follows:
\begin{align*}
\text{Goals} \qquad G &\defi \top \alt \conj{G_1}{G_2} \alt P \\
\text{Clauses} \qquad C &\defi \pi{\ctx}{P \from G} \\
\text{Programs} \qquad D &\defi \nil \alt D, C
\end{align*}
\note{Conjunction goals should probably not have a tree structure, i.e. be flat lists instead. I ignore this difference below.}

We impose on logic programs the following type system:

\begin{judgement}{\isGoal{\ctx}{G}}
{$G$ is a valid goal relative to $\ctx$ and $\sig$}
%
\begin{prooftree}
  \leftl{\rulename{G-Top} :}
  \ax{\isGoal{\ctx}{\top}}
\end{prooftree}

\begin{prooftree}
  \ninf{\isGoal{\ctx}{G_1}}
  \ninf{\isGoal{\ctx}{G_2}}
  \leftl{\rulename{G-Conj} :}
  \binf{\isGoal{\ctx}{\conj{G_1}{G_2}}}
\end{prooftree}

\begin{prooftree}
  \ninf{\hasKind{\ctx}{P}{\type}}
  \leftl{\rulename{G-Prim} :}
  \uinf{\isGoal{\ctx}{P}}
\end{prooftree}
%
\end{judgement}

\begin{judgement}{\isClause{C}}
{$C$ is valid clause relative to $\sig$}
%
\begin{prooftree}
  \ninf{$\typing{c}{(\Pi \, \ctx_1 . \, P_1 \to \ldots \to \Pi \, \ctx_n . \, P_n \to \Pi \, \ctx_0 . \, P_0)} \in \sig$}
  \leftl{\rulename{C-Clause} :}
  \uinf{\isClause{\pi{(\ctx_1, \ldots, \ctx_n, \ctx_0)}{P_0 \from P_n \wedge \ldots \wedge P_1}}} % note that (P_n \wedge \ldots \wedge P_1) is automatically a valid goal wrt \ctx
\end{prooftree}
%
\end{judgement}

\begin{judgement}{\isProg{D}}
{$D$ is a valid program relative to $\sig$}
%
\begin{prooftree}
  \leftl{\rulename{D-Nil} :}
  \ax{\isProg{\nil}}
\end{prooftree}

\begin{prooftree}
  \ninf{\isProg{D}}
  \ninf{\isClause{C}}
  \leftl{\rulename{D-Cons} :}
  \binf{\isProg{D, C}}
\end{prooftree}
%
\end{judgement}

In LF syntax, the non-dependent function type $A \to A'$ is syntactic sugar for $\pi{\typing{x}{A}}{A'}$, for some $x$ with $x \notin \FV{A'}$.
% And indeed, the pretty printer of Twelf's type reconstruction performs this transformation in all cases.
So the premise in rule \rulename{C-Clause} makes no demand on the shape of the type; the rule is applicable for all $\typing{c}{A} \in \sig$ (for a valid $\sig$).

We can now present an operational semantics on valid logic programs.

% discuss invariant: G[\sub \circ \subvar \circ grounding-sub] ``true''

\begin{judgement}{\solveGoal{\sub}{G}{\subvar}}
{$G$ under $\sub$ has solution substitution $\subvar$ in $D$}
%
\begin{prooftree}
  \leftl{\rulename{E-Top} :}
  \ax{\solveGoal{\sub}{\top}{\idsub}}
\end{prooftree}

\begin{prooftree}
  \ninf{\solveGoal{\sub}{G_1}{\subvar_1}}
  \ninf{\solveGoal{\sub \circ \subvar_1}{G_2}{\subvar_2}}
  \leftl{\rulename{E-Conj} :}
  \binf{\solveGoal{\sub}{\conj{G_1}{G_2}}{\subvar_1 \circ \subvar_2}}
\end{prooftree}

%\begin{prooftree}
  \ninf{$(\pi{\ctx}{Q \from G}) \in D$}
  \ninf{\mguParts{P[\sub], Q}{\subvar_1}{\subvar_1'}}
  \ninf{\solveGoal{\subvar_1'}{G}{\subvar_2}}
  \leftl{\rulename{E-Prim} :}
  \rightl{($\dom{\ctx} \cap \FV{P[\sub]} = \emptyset$)}
  \tinf{\solveGoal{\sub}{P}{\subvar_1 \circ \subvar_2}}
  
  \hspace{-75pt} \DisplayProof % temp solution, use boxes?
%\end{prooftree}
%
\end{judgement}

% note that the disjointness requirement can always be satisfied by renaming \ctx

Since we are using concrete substitutions, we wish to keep track of type information -- to keep them valid, and in order for composition to be a well-defined operation.
In particular, we maintain the following invariant for the judgement $\solveGoal{\sub}{G}{\subvar}$.
Given valid contexts $\ctxvar$ and $\ctxvar'$, if $\subHasCtx{\ctxvar'}{\sub}{\ctxvar}$ and $\isGoal{\ctxvar}{G}$, then $\subHasCtx{\ctxvar''}{\subvar}{\ctxvar'}$ (for some $\ctxvar''$).
\note{Should I make this ``invariant'' explicit in the judgement, i.e. have objects like substitutions carry along the contexts by which they are valid?}

The identity substitution in rule \rulename{E-Top}, then, should really be $\idsub_{\ctxvar'}$.
And given valid $\ctx$ and $\ctxvar$ (with disjoint domains), the judgement ${\mguParts{P, Q}{\subvar}{\subvar'}}$ % {} to prevent line break
is defined to hold if $\hasKind{\ctxvar}{P}{\type}$, $\hasKind{\ctx}{Q}{\type}$, $\subHasCtx{\ctxvar'}{\subvar}{\ctxvar}$, $\subHasCtx{\ctxvar'}{\subvar'}{\ctx}$, $\subHasCtx{\ctxvar'}{(\subvar, \subvar')}{(\ctxvar, \ctx)}$, and $[(\subvar, \subvar')] \in \mgu{\ueqn{P}{Q}}$ -- for some $\ctxvar'$.
We note that $\mguParts{P, Q}{\subvar}{\subvar'}$ in particular implies $P[\subvar] = P[\subvar, \subvar'] = Q[\subvar, \subvar'] = Q[\subvar']$.
\note{I will of course need to discuss this further eventually, but the idea is to split an mgu into two constituent parts in service of the typing constraints. This is a safe operation; prompt me to explain.}

\section*{Mode Checking}

We define a mode system.

\begin{align*}
\text{Modes} \qquad \mode &\defi \+ \alt \- \alt \* \\
\text{Mode sequences} \qquad \modeseq &\defi \nil \alt \mode, \modeseq
\end{align*}

A \textit{mode signature} $\msig$ is a mapping from type constants to mode sequences.
It is valid with respect to a signature $\sig$ if, for all type constants $a$ defined in $\sig$, $\msig(a)$ is defined and has a length matching the number of arguments in the kind assigned to $a$ in $\sig$.
Note that we can safely set $\msig(a)$ to $(\*, \ldots, \*)$ (of the appropriate length) if it is otherwise undefined, hence we can extend a mode signature to match a signature.

We need a notion of approximate substitutions, which keep track of assumptions about variables being grounded.

\begin{align*}
\text{Approximate terms} \qquad \appterm &\defi \unk \alt \gnd \\
\text{Approximate substitutions} \qquad \appsub &\defi \emptysub \alt \appsub, \for{\appterm}{x}
\end{align*}

As usual, we require that any non-empty approximate substitution $(\appsub, \for{\appterm}{x})$ satisfies $x \neq y$ for all $\for{\appterm'}{y} \in \appsub$.

\begin{judgement}{\appsubMatch{\appsub}{\ctx}}
{$\appsub$ matches $\ctx$}
%
\begin{prooftree}
  \leftl{\rulename{A-Nil} :}
  \ax{\appsubMatch{\emptysub}{\nil}}
\end{prooftree}

\begin{prooftree}
  \ninf{\appsubMatch{\appsub}{\ctx}}
  \leftl{\rulename{A-Cons} :}
  \uinf{\appsubMatch{(\appsub, \for{\appterm}{x})}{(\ctx, \typing{x}{P})}}
\end{prooftree}
%
\end{judgement}

For any context $\ctx$, we let $\unk_{\ctx}$ be the approximate substitution that matches $\ctx$ and assigns $\unk$ to all variables.

We now define the notion of being ground relative to an abstract substitution:

\begin{judgement}{\isGround{\appsub}{M}}
{$M$ is ground relative to $\appsub$}
%
\begin{prooftree}
  \ninf{$\for{\gnd}{x} \in \appsub$}
  \leftl{\rulename{Gnd-Var} :}
  \uinf{\isGround{\appsub}{x}}
\end{prooftree}

\begin{prooftree}
  \leftl{\rulename{Gnd-Const} :}
  \ax{\isGround{\appsub}{c}}
\end{prooftree}

\begin{prooftree}
  \ninf{\isGround{\appsub}{N}}
  \ninf{\isGround{\appsub}{M}}
  \leftl{\rulename{Gnd-App} :}
  \binf{\isGround{\appsub}{\app{N}{M}}}
\end{prooftree}
%
\end{judgement}

We will only use the judgement $\isGround{\appsub}{M}$ when, for some context $\ctx$, $\appsubMatch{\appsub}{\ctx}$ and $\hasType{\ctx}{M}{A}$ (for some $A$).
Note that, for any $\ctx$ and $M$, the judgement $\isGround{\unk_{\ctx}}{M}$ corresponds to the regular ``$M$ is ground''.

We are now ready to define mode checking:

\begin{judgement}{\mcheckIn{\appsub}{P}}
{input variables in $P$ are ground in $\appsub$}
%
\begin{prooftree}
  \ninf{\mcheckInCorr{\appsub}{P}{\nil}}
  \leftl{\rulename{CheckIn} :}
  \uinf{\mcheckIn{\appsub}{P}}
\end{prooftree}
%
\end{judgement}

\begin{judgement}{\mcheckInCorr{\appsub}{P}{\modeseq}}
{input variables in $P$, according to $\modeseq$, are ground in $\appsub$}
%
\begin{prooftree}
  \leftl{\rulename{CheckIn-Const} :}
  \rightl{($\msig(a) = \modeseq$)}
  \ax{\mcheckInCorr{\appsub}{a}{\modeseq}}
\end{prooftree}

\begin{prooftree}
  \ninf{\mcheckInCorr{\appsub}{P}{(\mode, \modeseq)}}
  \leftl{\rulename{CheckIn-App1} :}
  \rightl{($\mode \neq \+$)}
  \uinf{\mcheckInCorr{\appsub}{\app{P}{M}}{\modeseq}}
\end{prooftree}

\begin{prooftree}
  \ninf{\mcheckInCorr{\appsub}{P}{(\+, \modeseq)}}
  \ninf{\isGround{\appsub}{M}}
  \leftl{\rulename{CheckIn-App2} :}
  \binf{\mcheckInCorr{\appsub}{\app{P}{M}}{\modeseq}}
\end{prooftree}
%
\end{judgement}

\begin{judgement}{\mcheckOut{\appsub}{P}}
{output variables in $P$ are ground in $\appsub$}
%
\begin{center}
Analogous, with $\+$ replaced by $\-$.
\end{center}
%
\end{judgement}

We require for the judgements $\mcheckIn{\appsub}{P}$ and $\mcheckOut{\appsub}{P}$ that, for some context $\ctx$, $\appsubMatch{\appsub}{\ctx}$ and $\hasKind{\ctx}{P}{\type}$.

\begin{judgement}{\maddIn{\appsub}{P}{\appsub'}}
{$\appsub'$ is $\appsub$ with input variables in $P$ made ground}
%
\begin{prooftree}
  \ninf{\maddInCorr{\appsub}{P}{\nil}{\appsub'}}
  \leftl{\rulename{AddIn} :}
  \uinf{\maddIn{\appsub}{P}{\appsub'}}
\end{prooftree}
%
\end{judgement}

\begin{judgement}{\maddInCorr{\appsub}{P}{\modeseq}{\appsub'}}
{$\appsub'$ is $\appsub$ with input variables in $P$, according to $\modeseq$, made ground}
%
\begin{prooftree}
  \leftl{\rulename{AddIn-Const} :}
  \rightl{($\msig(a) = \modeseq$)}
  \ax{\maddInCorr{\appsub}{a}{\modeseq}{\appsub}}
\end{prooftree}

\begin{prooftree}
  \ninf{\maddInCorr{\appsub}{P}{(\mode, \modeseq)}{\appsub'}}
  \leftl{\rulename{AddIn-App1} :}
  \rightl{($\mode \neq \+$)}
  \uinf{\maddInCorr{\appsub}{\app{P}{M}}{\modeseq}{\appsub'}}
\end{prooftree}

\begin{prooftree}
  \ninf{\maddInCorr{\appsub}{P}{(\+, \modeseq)}{\appsub'}}
  \leftl{\rulename{AddIn-App2} :}
  \uinf{\maddInCorr{\appsub}{\app{P}{M}}{\modeseq}{\appsub'[\for{\gnd}{\FV{M}}]}}
\end{prooftree}
%
\end{judgement}

In rule \rulename{AddIn-App2}, $\appsub'[\for{\gnd}{\FV{M}}]$ denotes replacing, for all $x \in \FV{M}$, $\for{\appterm}{x}$ with $\for{\gnd}{x}$ in $\appterm'$.

\begin{judgement}{\maddOut{\appsub}{P}{\appsub'}}
{$\appsub'$ is $\appsub$ with output variables in $P$ made ground}
%
\begin{center}
Analogous, with $\+$ replaced by $\-$.
\end{center}
%
\end{judgement}

We require for the judgements $\maddIn{\appsub}{P}{\appsub'}$ and $\maddOut{\appsub}{P}{\appsub'}$ that, for some context $\ctx$, $\appsubMatch{\appsub}{\ctx}$, $\appsubMatch{\appsub'}{\ctx}$, and $\hasKind{\ctx}{P}{\type}$.

\begin{judgement}{\mcheckGoal{\appsub}{G}{\appsub'}}
{$G$ is mode correct from $\appsub$ to $\appsub'$ relative to $\msig$}
%
\begin{prooftree}
  \leftl{\rulename{MG-Top} :}
  \ax{\mcheckGoal{\appsub}{\top}{\appsub}}
\end{prooftree}

\begin{prooftree}
  \ninf{\mcheckGoal{\appsub}{G_1}{\appsub''}}
  \ninf{\mcheckGoal{\appsub''}{G_2}{\appsub'}}
  \leftl{\rulename{MG-Conj} :}
  \binf{\mcheckGoal{\appsub}{\conj{G_1}{G_2}}{\appsub'}}
\end{prooftree}

\begin{prooftree}
  \ninf{\mcheckIn{\appsub}{P}}
  \ninf{\maddOut{\appsub}{P}{\appsub'}}
  \leftl{\rulename{MG-Prim} :}
  \binf{\mcheckGoal{\appsub}{P}{\appsub'}}
\end{prooftree}
%
\end{judgement}

We require for the judgement $\mcheckGoal{\appsub}{G}{\appsub'}$ that, for some context $\ctx$, $\appsubMatch{\appsub}{\ctx}$, $\appsubMatch{\appsub'}{\ctx}$, and $\isGoal{\ctx}{G}$.

\begin{judgement}{\mcheckClause{C}}
{$C$ is mode correct relative to $\msig$}
%
\begin{prooftree}
  \ninf{\maddIn{\unk_{\ctx}}{P}{\appsub_1}}
  \ninf{\mcheckGoal{\appsub_1}{G}{\appsub_2}}
  \ninf{\mcheckOut{\appsub_2}{P}}
  \leftl{\rulename{MC-Clause} :}
  \tinf{\mcheckClause{\pi{\ctx}{P \from G}}}
\end{prooftree}
%
\end{judgement}

\begin{judgement}{\mcheckProg{D}}
{$D$ is mode correct relative to $\msig$}
%
\begin{prooftree}
  \leftl{\rulename{MD-Nil} :}
  \ax{\mcheckProg{\nil}}
\end{prooftree}

\begin{prooftree}
  \ninf{\mcheckProg{D}}
  \ninf{\mcheckClause{C}}
  \leftl{\rulename{MD-Cons} :}
  \binf{\mcheckProg{D, C}}
\end{prooftree}
%
\end{judgement}

Finally, we need to relate approximate and concrete substitutions.
Informally, we say an approximate substitution $\appsub$ \textit{approximates} a concrete substitution $\sub$ if they match the same domain $\ctx$ and all variables marked $\gnd$ by $\appsub$ are made ground by $\sub$.

\begin{judgement}{$\appsub \approx \sub$}
{$\appsub$ approximates $\sub$}
%
\begin{prooftree}
  \leftl{\rulename{Approx-Nil} :}
  \ax{$\emptysub \approx \emptysub$}
\end{prooftree}

\begin{prooftree}
  \ninf{$\appsub \approx \sub$}
  \leftl{\rulename{Approx-Unk} :}
  \uinf{$(\appsub, \for{\unk}{x}) \approx (\sub, \for{M}{x})$}
\end{prooftree}

\begin{prooftree}
  \ninf{$\appsub \approx \sub$}
  \ninf{\isGround{\unk}{M}}
  \leftl{\rulename{Approx-Gnd} :}
  \binf{$(\appsub, \for{\gnd}{x}) \approx (\sub, \for{M}{x})$}
\end{prooftree}
%
\end{judgement}

We will only use the judgement $\appsub \approx \sub$ when, for some context $\ctx$, $\appsubMatch{\appsub}{\ctx}$ and $\subHasCtx{\ctx'}{\sub}{\ctx}$ (for some $\ctx'$).
In the second premise of rule \rulename{Approx-Gnd}, $\unk$ then refers to the approximate substitution $\unk_{\ctx'}$.

\begin{lemma}
\label{lem:mode-soundness}
Let $\sig$ be a valid signature, $\msig$ a valid mode signature relative to $\sig$, $D$ a logic program with $\isProg{D}$ and $\mcheckProg{D}$.
Let $\ctxvar$ be a valid context, $G$ a goal with $\isGoal{\ctxvar}{G}$, $\appsub$ an approximate substitution with $\appsubMatch{\appsub}{\ctxvar}$, and $\sub$ a substitution with $\subHasCtx{\ctxvar'}{\sub}{\ctxvar}$ (for some $\ctxvar'$).

If $\solveGoal{\sub}{G}{\subvar}$ (by $\E$, for some $\subvar$) and $\mcheckGoal{\appsub}{G}{\appsub'}$ (by $\M$, for some $\appsub'$) and $\appsub \approx \sub$, then $\appsub' \approx (\sub \circ \subvar)$.
\end{lemma}

\begin{proof}
By induction on $\E$.

\begin{itemize}
	\item Case \rulename{E-Top} \\
  We have
  %
  \begin{prooftree}
    \leftl{$\E =$}
    \ax{\solveGoal{\sub}{\top}{\idsub_{\ctxvar'}}}
  \end{prooftree}
  and so
  \begin{prooftree}
    \leftl{$\M =$}
    \ax{\mcheckGoal{\appsub}{\top}{\appsub}}
  \end{prooftree}
  %
  But then we have $\appsub \approx (\sub \circ \idsub_{\ctxvar'}) = \sub$ by assumption.
  
  \item Case \rulename{E-Conj} \\
  We have
  %
  \begin{prooftree}
    \prem{\E_1}{\solveGoal{\sub}{G_1}{\subvar_1}}
    \prem{\E_2}{\solveGoal{\sub \circ \subvar_1}{G_2}{\subvar_2}}
    \leftl{$\E =$}
    \binf{\solveGoal{\sub}{\conj{G_1}{G_2}}{\subvar_1 \circ \subvar_2}}
  \end{prooftree}
  and so
  \begin{prooftree}
    \prem{\M_1}{\mcheckGoal{\appsub}{G_1}{\appsub_1}}
    \prem{\M_2}{\mcheckGoal{\appsub_1}{G_2}{\appsub_2}}
    \leftl{$\M =$}
    \binf{\mcheckGoal{\appsub}{\conj{G_1}{G_2}}{\appsub_2}}
  \end{prooftree}
  %
  By the IH on $\E_1$ with $\M_1$, we get $\appsub_1 \approx (\sub \circ \subvar_1)$.
  Then by the IH on $\E_2$ with $\M_2$, we get $\appsub_2 \approx ((\sub \circ \subvar_1) \circ \subvar_2)$.
  But since composition of concrete substitutions is associative, we have the required $\appsub_2 \approx (\sub \circ (\subvar_1 \circ \subvar_2))$.
  
  \item Case \rulename{E-Prim} \\
  We have
  %
  \begin{prooftree}
    \ninf{$(\pi{\ctx}{Q \from G}) \in D$}
    \ninf{\mguParts{P[\sub], Q}{\subvar_1}{\subvar_1'}}
    \prem{\E_1}{\solveGoal{\subvar_1'}{G}{\subvar_2}}
    \leftl{$\E =$}
    \tinf{\solveGoal{\sub}{P}{\subvar_1 \circ \subvar_2}}
  \end{prooftree}
  and so
  \begin{prooftree}
    \ninf{\mcheckIn{\appsub}{P}}
    \ninf{\maddOut{\appsub}{P}{\appsub'}}
    \leftl{$\M =$}
    \binf{\mcheckGoal{\appsub}{P}{\appsub'}}
  \end{prooftree}
  %
  Since $(\pi{\ctx}{G \to Q}) \in D$ and $\mcheckProg{D}$, we also have a derivation
  \begin{prooftree}
    \ninf{\maddIn{\unk_{\ctx}}{Q}{\appsub_1}}
    \prem{\M_1}{\mcheckGoal{\appsub_1}{G}{\appsub_2}}
    \ninf{\mcheckOut{\appsub_2}{Q}}
    \tinf{\mcheckClause{\pi{\ctx}{G \to Q}}}
  \end{prooftree}
  %
  Since $\mcheckIn{\appsub}{P}$ and $\appsub \approx \sub$ (by assumption), $\sub$ -- and in turn $\sub \circ \subvar_1$ -- grounds the input variables in $P$.
  Hence the input variables in $Q$ are grounded by $\subvar_1'$.
  \note{I should probably formulate this as a general lemma.}
  That is, by definition of $\appsub_1$, $\appsub_1 \approx \subvar_1'$.
  
  Then by the IH on $\E_1$ with $\M_1$, we get $\appsub_2 \approx (\subvar_1' \circ \subvar_2)$.
  So by $\mcheckOut{\appsub_2}{Q}$, the output variables in $Q$ are grounded by $\subvar_1' \circ \subvar_2$.
  We note $Q[\subvar_1' \circ \subvar_2] = (Q[\subvar_1'])[\subvar_2] = (P[\sub][\subvar_1])[\subvar_2] = P[\sub \circ (\subvar_1 \circ \subvar_2)]$.
  \note{This holds only because of the invariants, i.e. because the substitutions are valid relative to ``matching'' contexts.}
  And so the output variables in $P$ are grounded by $\sub \circ (\subvar_1 \circ \subvar_2)$.
  And, of course, $\appsub \approx \sub$ implies $\appsub \approx (\sub \circ (\subvar_1 \circ \subvar_2))$.
  Hence, by definition of $\appsub'$, $\appsub' \approx (\sub \circ (\subvar_1 \circ \subvar_2))$ as required.
  %
\end{itemize}
\end{proof}

\begin{theorem}
\label{thm:mode-soundness}
Let $\sig$ be a valid signature, $\msig$ a valid mode signature relative to $\sig$, $D$ a logic program with $\isProg{D}$ and $\mcheckProg{D}$.
Let $\ctx$ be a valid context and let $P = a \; M_1 \; \ldots \; M_n$ satisfy $\hasKind{\ctx}{P}{\type}$.
Let $\msig(a) = (m_1, \ldots, m_n)$ and suppose, for all $i$, that $m_i = \+$ implies $M_i$ ground.

If $\solveGoal{\idsub_{\ctx}}{P}{\subvar}$, then, for all $i$, $m_i = \-$ implies $M_i[\subvar]$ ground.
\end{theorem}

\begin{proof}
Immediate from Lemma~\ref{lem:mode-soundness} (taking $\appsub = \unk_{\ctx}$).
\end{proof}

%%% end
