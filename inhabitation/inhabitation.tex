\clearpage

\section*{Inhabitation}

\begin{definition}[Inhabitation]
\label{def:inhabitation}
Let $\ctx$ be a valid context, $P$ a primitive type such that $\hasKind{\ctx}{P}{\type}$.
$P$ is \emph{(strongly) inhabited} relative to $\ctx$ if, for all substitutions $\sub$ with $\subHasCtx{\nil}{\sub}{\ctx}$, there exists an object $M$ such that $\hasType{\nil}{M}{P[\sub]}$.
It is \emph{weakly inhabited} or \emph{inhabitable} if there exist both a substitution $\sub$ with $\subHasCtx{\nil}{\sub}{\ctx}$ and an object $M$ such that $\hasType{\nil}{M}{P[\sub]}$.
\end{definition}

Note that the two definitions coincide when the context $\ctx$ is empty and $P$ is ground; the empty substitution $\emptysub$ satisfies $\subHasCtx{\nil}{\emptysub}{\nil}$ and $P[\emptysub] = P$, so both definitions require the existence of a ground object $M$ for which $\hasType{\nil}{M}{P}$.
(However, strong inhabitation does not always imply weak inhabitation.)
% illustrate this note with an example

We now define a converse notion to inhabitability.

\begin{definition}[Emptiness]
\label{def:emptiness}
Let $\ctx$ be a valid context, $P$ a primitive type such that $\hasKind{\ctx}{P}{\type}$.
$P$ is \emph{empty} or \emph{uninhabitable} if, for all substitutions $\sub$ with $\subHasCtx{\nil}{\sub}{\ctx}$, there does not exist an object $M$ such that $\hasType{\nil}{M}{P[\sub]}$.
Or equivalently, $P$ is empty if there exists no substitution $\sub$ with $\subHasCtx{\nil}{\sub}{(\ctx, \typing{x}{P})}$.
Or equivalently again, $P$ is empty if the coverage goal $\goal{(\ctx, \typing{x}{P})}{x}{P}$ is covered by the empty set of patterns, i.e. it has no ground instances.
\end{definition}

Since emptiness is exactly the converse of inhabitability, the notion is sound in the following sense:
A type cannot be both empty and inhabitable.
And, in particular, a ground type $P$ (with $\hasKind{\nil}{P}{\type}$) cannot be both empty and inhabited.

\begin{theorem}[Conservativity of emptiness]
\label{thm:conservativity-emptiness}
Let $\ctx$ be a valid context, $P$ a primitive type such that $\hasKind{\ctx}{P}{\type}$, $\S$ a complete set of substitutions for $(\ctx, \typing{x}{P})$.
% assuming $x \notin \dom{\ctx}..
Then the goal $\goal{(\ctx, \typing{x}{P})}{x}{P}$ has no ground instances if all members of $\set{\goal{\ctx''}{x[\sub]}{P[\sub]} \| \triple{\ctx''}{\sub}{\ctx'} \in \S}$ have no ground instances.
% and only if
\end{theorem}

\begin{proof}
Since $\S$ is complete for $(\ctx, \typing{x}{P})$, the set of ground instances of $\goal{(\ctx, \typing{x}{P})}{x}{P}$ is a subset of the union of the sets of ground instances of $\goal{\ctx''}{x[\sub]}{P[\sub]}$ for each $\triple{\ctx''}{\sub}{\ctx'} \in \S$.
\end{proof}

% By Theorem~\ref{thm:splitting-props}, then, applying the splitting operation preserves the question of emptiness.
% That is, in trying to decide whether a type $\hasKind{\ctx}{A}{\type}$ is empty it is safe to instead consider the result of splitting the goal $\hasKind{(\ctx, \typing{x}{A})}{x}{A}$ any number of times.
% Of course, an algorithm (necessarily incomplete) that attempts to decide emptiness by repeated splitting must have a termination criterion.

%[
%Introduce the subordination relation, and propose an algorithm that stops splitting when the type of a new variable is not strictly subordinate (to the type of the splitting variable).
%Argue that the algorithm terminates.
%And argue that, although necessarily incomplete, it is still 'useful'.
%]

%%% end
