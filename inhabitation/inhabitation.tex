% \clearpage

\section*{Inhabitation Problem}

A type $\hasKind{\ctx}{A}{\type}$ is \textit{(strongly) inhabited} if, for all substitutions $\subHasCtx{\nil}{\sub}{\ctx}$, there exists an object $M$ such that $\hasType{\nil}{M}{\subst{A}{\sub}}$.
It is \textit{weakly inhabited} or \textit{inhabitable} if there exist both a substitution $\subHasCtx{\nil}{\sub}{\ctx}$ and an object $M$ such that $\hasType{\nil}{M}{\subst{A}{\sub}}$.
Note that the two definitions coincide when the context $\ctx$ is empty and $A$ is a closed type; the empty substitution $\emptysub$ satisfies $\subHasCtx{\nil}{\emptysub}{\nil}$ and $\subst{A}{\emptysub} = A$, so both definitions require the existence of a closed object $M$ for which $\hasType{\nil}{M}{A}$.
(However, strong inhabitation does not always imply weak inhabitation.)

As a converse notion, we say a type $\hasKind{\ctx}{A}{\type}$ is \textit{empty} or \textit{uninhabitable} if, for all substitutions $\subHasCtx{\nil}{\sub}{\ctx}$, there does not exist an object $M$ such that $\hasType{\nil}{M}{\subst{A}{\sub}}$.
Or equivalently, a type $\hasKind{\ctx}{A}{\type}$ is empty if there exists no substitution $\subHasCtx{\nil}{\sub}{(\ctx, \typing{x}{A})}$.
Or equivalently again, $\hasKind{\ctx}{A}{\type}$ is empty if the coverage goal $\hasKind{(\ctx, \typing{x}{A})}{x}{A}$ is covered by the empty set of patterns, i.e. it has no ground instances.

Since emptiness is exactly the converse of inhabitability, the notion is sound in the following sense:
A type cannot be both empty and inhabitable.
And, in particular, a closed type $\hasKind{\nil}{A}{\type}$ cannot be both empty and inhabited.

\begin{theorem}[Conservativity of emptiness]
Let $\hasKind{\ctx}{A}{\type}$ be a type and $\S = \set{\subHasCtx{\ctx_i}{\sub_i}{(\ctx, \typing{x}{A})} \;\;|\;\; i = 1, ..., n}$ a non-redundant, complete set of substitutions.
Then $\hasKind{(\ctx, \typing{x}{A})}{x}{A}$ has no ground instances if and only if all members of $\set{\hasKind{\ctx_i}{\subst{x}{\sub_i}}{\subst{A}{\sub_i}} \;\;|\;\; i = 1, ..., n}$ have no ground instances.
\end{theorem}

\begin{proof}
Since $\S$ is complete, the sets of ground instances are exactly the same.
\end{proof}

By Theorem~\ref{thm:splitting-props}, then, applying the splitting operation preserves the question of emptiness.
That is, in trying to decide whether a type $\hasKind{\ctx}{A}{\type}$ is empty it is safe to instead consider the result of splitting the goal $\hasKind{(\ctx, \typing{x}{A})}{x}{A}$ any number of times.
Of course, an algorithm (necessarily incomplete) that attempts to decide emptiness by repeated splitting must have a termination criterion.

[
Introduce the subordination relation, and propose an algorithm that stops splitting when the type of a new variable is not strictly subordinate (to the type of the splitting variable).
Argue that the algorithm terminates.
And argue that, although necessarily incomplete, it is still 'useful'.
]
